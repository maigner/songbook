\beginsong{Geld oder Leben}[by={E.A.V.}, index={Juhuu}]

\beginverse
Es beherrscht der Obolus seit jeher unsern Globulus.
Mit anderen Worten: Der Planet sich primär um das eine dreht!
Drum: Schaffe, schaffe, Häusle baue! Butterbrot statt Schnitzel kaue! 
Denn wer nicht den Pfennig ehrt, der wird nie ein Dagobert!
\endverse

\beginchorus
Geld, Geld -- oder Leben!
Geld, Geld -- oder Leben!
Geld, Geld -- oder Leben!
Geld, Geld -- Geld oder Leben!
\endchorus

\beginverse
Ach, ach was!
Es ist vom Volksmund eine Linke,
daß das Geld gar übel stinke.
Wahr ist vielmehr: Ohne Zaster
beißt der Mensch ins Straßenpflaster.

Geld, Geld ...
\endverse

\beginverse
Es sagt das Sprichwort: „Spare, spare,
denn dann hast du in der Not!“
Der eine spart, kriegt graue Haare,
der and're erbt nach seinem Tod.
\endverse
\beginverse
Dollar, D-Mark, Schilling, Lire,
Rubel, Franken oder Pfund:
Die Vermehrung uns'rer Währung
ist der wahre Lebensgrund.
\endverse
\beginverse
Der Mammon sagt, man, sei ein schnöder,
doch ohne ihn ist's noch viel öder.
Im Westen, Osten oder Süden
überleben nur die Liquiden.
\endverse
\beginverse
Ohne Rubel geht die Olga
mit dem Iwan in die Wolga.
Für Karl-Otto gilt dasselbe:
Ohne Deutschmark in die Elbe!

Geld, Geld...
\endverse

\beginverse
Wenn Achmed keine Drachmen hat,
lutscht traurig er am Dattelblatt.
Es macht Umberto ohne Lire
mit Spaghetti Harakiri.
\endverse
\beginverse
Hat der Svensson keine Öre,
eilt von dannen seine Göre.
Nimmt man mir den letzten Schilling,
hab' auch ich kein gutes Feeling.
\endverse

\beginchorus
Geld, Geld...
\endchorus


\endsong