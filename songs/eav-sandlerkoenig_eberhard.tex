
\beginsong{Sandlerkönig Eberhard}[by={EAV}]
%\gtab{G7}{320001}

\chordsoff

\beginverse
\nolyrics C-DUR
\endverse


\beginverse
Ein wahrer Musterknabe war der Eberhard,
nach Schwiegermutterart.
Im Kirchenchor und als Student stieg er steil empor,
bis er sein Herz verlor!

Ihr Name, der war Julia, sie brach ihm das Herz.
Doch als sie ihn dann verließ, warf er sein junges Leben abgrundwärts!
\endverse


\beginverse
Schon bald sah man den Eberhard,
das Auge rot, die Leber hart,
immer tiefer in die Gosse sinken.
Sein Äußeres war dubios,
arbeits- und auch obdachlos
war er und fing schon langsam an zu stinken.
\endverse


\beginverse
Doch ganz egal, wie tief er fiel,
der Eberhard verfiel mit Stil,
er war ein Sandler ganz besond'rer Art.
Der einzige vom Südbahnhof,
der statt Fusel Glühwein soff,
das war der Sandlerkönig Eberhard!
\endverse

\beginverse
Legt er im Park sich nachts zur Ruh,
deckt er sich mit dem ``Spiegel'' zu
und traurig denkt er an die Zeit zurück.
Er schaut sich das Foto an,
des er kaum noch halten kann.
Die Julia, die war sein ganzes Glück!
\endverse


\beginchorus
Er war der Sandlerkönig, er war wie der Wein,
ein Vagabondo del amor, so echt und rein.
Er war der Sandlerkönig, er war wie der Wein,
doch wie bei Romeo und Julia - es hod net soll'n sein!
\endchorus

\beginverse
Der Sandlerkönig Eberhard
macht vor dem Tresen an Spagat,
da sieht er plötzlich eine Sandlerin.
Obwohl sie nicht nach Flieder riecht,
der Eberhard gleich niederbricht.
Es zieht ihn einfach magisch zu ihr hin!
\endverse

\beginchorus
Er war der Sandlerkönig...
\endchorus

\beginverse
Er sagt zur ihr: ``Pardon, Madam,
könnt i a Zigarett'n ham?''
und er schenkt ihr einen tiefen Blick.
Auf einmal schreit er: ``Jessas na!
Meiner Seel - die Julia!''
Es ist die Liebe auf den letzten Tschick!
\endverse


\beginchorus
Er war der Sandlerkönig...
\endchorus


\beginverse
Die beiden soffen Hand in Hand
im Glücksrausch alles durcheinand,
Fusel, Spiritus und Methanol.
Doch die Feier währt' nur kurz,
die Juli kriagt an Lebersturz,
rülpst und sagt dem Dasein ``Lebe wohl''!
\endverse

\beginverse
Der Eberhard rief: ``Liebste Mein!
Bist du nicht, will auch ich nicht sein!''
und nimmt den Todessaft aus ihrer Hand.
Weil ihm im Leben nichts mehr bleibt,
hat er sich mit dem Rest entleibt.
Wos was i, vielleicht san's jetzt beinand?
\endverse

\beginchorus
Er war der Sandlerkönig...
\endchorus

\endsong