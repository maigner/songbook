
\beginsong{Zwickts Mi}[by={Wolfgang Ambros}]

\gtab{E&}{6:X13331}

\chordson

\beginverse
Intro: G
\endverse       


\beginverse
 \[G]Gestern fahr i \[F]mit da Tramway R\[D]ichtung Favor\[G]iten.
Draußen rengts und dr\[F]innen stinkts und \[D]i steh in da m\[G]ittn.
Die L\[C]eid obs sitzn oder stengan, \[E&]olle homs des fade Aug.
Und s\[A]icha ned nur in da Tramway, i g\[D]laub des homs in gonzn Tog.\[G]
\endverse



\beginverse
Im \[^]Wirtshaus triff i \[^]immer an, der w\[^]as Gottwos daz\[^]ölt.
Er is so reich, er \[^]is so guat, er k\[^]ennt die gonze W\[^]öt.
In W\[^]irklichkeit is er a Sandler, h\[^]ocknstad und dauernd fett.
Des l\[^]etzte Weh in meine Augn, n\[^]a, i pock eam ned!
\endverse



\beginchorus
Zw\[G]ickts mi, i man i dr\[Am]am
\[F]Des derf net wor sein, wo s\[G]amma daham.
Zw\[G]ickts mi, ganz wurscht woh\[Am]in.
\[F]I kanns net glaubn, ob i \[D]ogsoffn bin.
\[G]Oba i glaub da hüft ka Zw\[Am]ickn
K\[A]ennt ma net vielleicht irgendwer  \[D]ane pickn.
D\[G]anke, jetzt is ma kl\[Am]ar,   \[Cm]es is w\[G]ar, \[D]es is w\[G]ar.
\endchorus

\beginverse
\nolyrics Pfeifen: \[G]  \[Am]   \[F]   \[G]
\endverse


\beginverse
Die J\[^]ugend hat kein \[^]Ideal, kann S\[^]inn für wahre W\[^]erte
Den jungen Leuten g\[^]eht's zu gut, sie k\[^]ennen keine H\[^]ärte
So r\[^]eden die, die nur in Oarsch kreun, Schm\[^]iergeld nehmen, packeln dan,
n\[^]ach an Skandal dann pensioniert wern, k\[^]urz a echtes Vorbild san.
\endverse


\beginchorus
2x

Zw\[G]ickts mi, i man i dr\[Am]am
\[F]Des derf net wor sein, wo s\[G]amma daham.
Zw\[G]ickts mi, ganz wurscht woh\[Am]in.
\[F]I kanns net glaubn, ob i \[D]ogsoffn bin.
\[G]Oba i glaub da hüft ka Zw\[Am]ickn
K\[A]ennt ma net vielleicht irgendwer  \[D]ane pickn.
D\[G]anke, jetzt is ma kl\[Am]ar,   \[Cm]es is w\[G]ar, \[D]es is w\[G]ar.
\endchorus

\endsong


