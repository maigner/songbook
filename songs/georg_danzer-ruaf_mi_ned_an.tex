\beginsong{Ruaf mi ned an}[by={Georg Danzer}]

%\gtab{G7}{320001}

\chordson
\transpose{10}

%\gtab{Bm9}{7:131113}
%\gtab{A/B}{9:14312X}
%\gtab{Em}{7:XX3321}
%\gtab{F#7}{242322}

\beginverse
\nolyrics Intro: \[D A Em G A D] 
\endverse

\beginverse\memorize


R\[D]uaf mi net an weu du w\[A]ast doch genau das i n\[Em]immer mehr wü

und a n\[G]immer mehr k\[A]au, bitte r\[D]uaf net an.

Ruaf mi net an weu i h\[A]ea nua dei Stimm und dann schl\[Em]of i net ei

bis i w\[G]ieder die bl\[A]edn Tabl\[D]etten nimm.

W\[G]eit host mi brocht, i steh a\[F#m]uf in da Nocht und dann g\[Em]eh i spazieren. \[D]

G\[G]anz ohne Grund, i hob n\[F#m]ed a moi an Hund zum \[A]äußerln fian.
     
\endverse

\beginverse
^Und wann i ham kum is ^ollas wias woa, und mei P^olster riacht

immer no n^och deine H^oa heast i w^ia a Noa.

I was du host jetzt an Fr^eind mit an Porsche, s^og eam doch

er soi in ^Orsch geh, und k^umm wida h^am zu mir.
\endverse

\beginverse
^Er geht mit dir jeden ^Abend fein essen, sog h^ost schon vergessen

wia a L^eberkas schm^eckt aus’n Z^eitungspapier.

\[G]Er fiat di aus ins The\[F#m]ater, des brennt eahm sei V\[Em]ater der Dillo \[D]

dab\[G]ei is a schmähstaht und sch\[F#m]iach und blad mit seine h\[A]undert Kilo.

\endverse


\beginverse

R^uaf mi net an weu du w^ast doch genau wo i w^ohn, wannst wos

wüst trau di h^er wannst ned z’f^eig dazu b^ist.

Ruaf mi net an weu du wast doch genau wo i wohn, wannst wos

wüst trau di her wannst ned z’feig dazu bist.






\endverse






\endsong
