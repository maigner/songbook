
\beginsong{Strada Del Sole}[by={Reinhard Fendrich}]
\gtab{h}{X2443X}

\chordson

\beginverse
\nolyrics Intro: G-DUR  
Wechselschlag: [d dud du ud dudu]
\endverse

\beginverse
\[G] I steh in da Hitz an da Strada del \[D]Sole.
\[G] Die Fiaß tamma weh in die neich'n Sand\[D]ale.
\[G] Mei Freindin is oposcht mit an Itali\[D]ano.
\[G ]Des Göld hams ma g'stessn, jetzt stehr i all\[D]an do,
und hab kane \[G]Lire. I hab kane \[D]Lire,
und kane Pap\[G]iere, so wos haut de net \[D]fire...
\endverse
\beginverse
\[h]Auf amoi woars \[e]päule \[C] mit dem Papag\[D]alle, mmm...
\[h]Und mi loßt's da \[e]anglahnt
\[C] in meine neich'n Sand\[D]ale, des is a Skand\[G]ale.
I hab kane \[D]Lire, und \[C]kane Pap\[G]iere, 
so wos haut di net \[D]fire.
\endverse
\beginverse
Er wollte Amore mit Bella Ragazza,
auf sentimentale und auf da Madrazza.
Dann is er no antanzt mi'n Alfa Romeo.
Z'erscht hab i'no ausglocht, und jetzt stehr i schee do,
und hab kane Lire. I hab kane Lire,
und kane Papiere, so wos haut de net fire...
\endverse
\beginverse
Er hot's mitn Schmäh packt auf dolce far niente, mmm...
net sehr vül im Hirn, ober molto potente, dem hau i die Zähnt ei!
I hab kane Lire, und kane Papiere, so wos haut di net fire.
\endverse
\beginverse
I wollt nach Firenze, nach Rom und nach Pisa,
doch jetzt hab i endgültig gnua von die Gfriesa.
Total abgebrannt steh i da ganz allani,
war i nur daham bliebn bei meine Kumpani.
\endverse
\beginverse
I winschat des ollas am liabsten zum Teifl, mmm...
was brauch i den Bledsinn,
I steh auf's Gänsehäufl, auf Italien pfeif i!
\endverse

\endsong