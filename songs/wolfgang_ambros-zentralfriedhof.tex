

\beginsong{Es lebe der Zentralfriedhof}[by={Wolfgang Ambros}]

\chordson

\beginverse
\nolyrics Intro: \[Em] \[Bm] \[Em] \[Bm]

\[G]Es lebe der Zentr\[Bm]alfriedhof, \[C]und alle seine \[G]Tote,
\[Em]Da Eintritt ist für l\[Am]ebende, heut’ \[D]ausnahmslos verb\[G]oten.
\[Em]Weil der Tod a \[Am]Fest heut gibt, die g\[D]anze lange N\[G]acht.
und \[C]von die Gäst ka \[Bm]einziger a \[Am]Eitrittskarten \[Em]braucht.
\endverse

\beginverse
Wann's Nocht wird über Simmering, kummt Leben in die Toten,
und drüb'n beim Krematorium tan's Knochenmork ohbrot'n.
Dort hinten bei der Marmorgruft, durt stengan zwa Skelette,
die stess'n mit zwa Urnen on und saufen um die Wette.
\endverse

\beginchorus
Am Zentr\[G]alfriedhof is St\[D]immung, weis seit L\[F]ebtag no net \[C]woa,
\[Am]weil alle Tot’n \[G]feiern heut seine \[D]ersten hundert Jahr.\[Em] \[Bm]
\endchorus

\beginverse
Es lebe der Zentralfriedhof, und seine Jubilare.
Sie lieg'n und sie verfeul'n scho durt seit über hundert Jahre.
Drauß't is' koit und drunt' is' worm, nur monchmol a bissel feucht,
A-wann ma so drunt' liegt, freut man sich, wenn's Grablaternderl
leucht'.
\endverse

\beginverse
Es lebe der Zentralfriedhof, die Szene wirkt makaber.
Die Pforrer tanz'n mit die Hur'n, und Juden mit Araber.
Heit san olle wieder lustich, heit lebt ollas auf,
im Mausoleum spü't a Band, die hot an Wohnsinnshammer d'rauf.
(Happy Birthday! Happy Birthday! Happy Birthday!)
\endverse

\beginchorus
Am Zentralfriedhof is' Stimmung, wia's sei Lebtoch no net wor,
weu olle Tot'n feiern heite seine erscht'n hundert Johr'.
(Happy Birthday! Happy Birthday!)
\endchorus

\beginverse
Es lebe der Zentralfriedhof, auf amoi mocht's an Schnoiza,
da Moser singt's Fiakerliad, und die Schrammeln spü'n an Woiza.
Auf amoi is' die Musi stü, und olle Augen glänz'n,
weu dort drü'm steht da Knoch'nmonn und winkt mit seiner Sens'n.
\endverse

\beginchorus
Am Zentralfriedhof is' Stimmung, wia's sei Lebtoch no net wor,
weu olle Tot'n feiern heite seine erscht'n hundert Johr'.
\endchorus

\endsong



