
\beginsong{Die Kinettn wo I schlof}[by={Wolfgang Ambros}]
%\gtab{G7}{320001}
\chordson

\beginverse
\nolyrics G-DUR Beginn auf \[D]
\endverse

%\chordsoff

\beginverse
Wann in da Fruah die Nocht
Gegnan Tog den kürzern ziagt
Und wenn da erste Sonnenstrah'
de letzte Dämmerung dawiagt,
dann woch i auf
in der Kinettn wo i schlof.
\endverse
\beginverse
Die Tschuschn kumman und i muaß mi
schleichn, sonst zagns mi an.
So kreul i halt ausse und putz ma
den Dreck o, so guat i kann.
So steh i auf, in der Kinettn wo i schlof.
\endverse
\beginverse
I hob mi scho seit zenn Tog nimmer rasiert
und nimmer gwoschn.
Und i hob nix als wie a Flaschn Rum
in da Mantltoschn.
De gib i ma zum Frühstück und dann
schnorr i an um a Zigarettn an -
und um an Schülling.
\endverse
\beginverse
Und de Leut kommen ma entgegn,
wie a Mauer kommens auf mi zua.
I bin da anzige der ihr entgegen geht
kummt ma vua -
Oba i reiß mi zamm und mach beim ersten Schritt
de Augn zua.
\endverse
\beginverse
Es is do ganz egal
ob i wos arbeit oder net,
wei fia de dünne Klostersuppn
genügts doch a wann i bet.
Laßts mi in Ruah
weu heit schüttns mei Kinettn zua.

Laßts mi in Ruah. 

\endverse


\endsong