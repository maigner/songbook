\beginsong{Die Blume aus dem Gemeindebau}[by={Wolfgang Ambros}]

\beginverse
\nolyrics
Intro: \[G] \[B7] \[Em] \[C7] \[G] \[D7] \[G]
\endverse


 

 
\beginchorus
Du bist die Bl\[G]ume aus \[B7]dem Gem\[Em]eindeb\[C]au,
ich w\[G]eiss ganz gen\[D7]au,
du b\[G]ist die richt'ge Fr\[C7]au für mich,
du Bl\[G]ume aus d\[D7]em Gemeindeb\[G]au.
\endchorus


\beginverse
Ohne d\[G]ich wär' d\[B7]ieser B\[Em]au so gr\[C]au,
und wer dich s\[G]ieht, sagt n\[D7]ur ``sch\[G]au, sch\[C7]au,
da geht die sch\[G]eenste Fr\[D7]au von Stadl\[G]au.''
\endverse

 
\beginverse
So wie du g\[Em]ehst, so wie du di bew\[C7]egst,
du w\[Em]asst gar net, wie sehr du mich err\[C7]egst,
and're h\[G]ab'n bei m\[B7]ir ka Ch\[Em]ance, \[C7]
auch wenn sie \[G]immer s\[D7]og'n ``Kummen'S F\[B7]ernseh'n, Herr Franz!''
I mecht von d\[Em]ir nur am\[D]oi a L\[G]ächeln kr\[C]iagn,
du scheenste Fr\[G]au von der V\[D]ierer-St\[G]iag'n.
\endverse

 
\beginchorus
Du bist die Bl\[G]ume aus \[B7]dem Gem\[Em]eindeb\[C]au,
deine \[G]Augen so bl\[D7]au,
wie \[G]ein Stadlauer Z\[C7]iegelteich,
du Bl\[G]ume aus d\[D7]em Gemeindeb\[G]au.


Und wann wer k\[G]ummat und s\[B7]ogat ``Na, wie w\[Em]är's, gnä' Fr\[C7]au?'',
dann kunnt 's leicht s\[G]ein, dass \[D7]i eam n\[G]iederh\[C7]au',
weu du bi\[G]st mei V\[D7]enus aus Stadl\[G]au.
\endchorus

 

\beginverse

\nolyrics
Solo
\[G]  \[B7]  \[Em]  \[C7]  \[G]  \[D7]  \[G] ...
\endverse
 

 
\beginverse
Wann i di s\[Em]iech, dann spüt's Gran\[C7]ada bei mir,
i k\[Em]ann nur sog'n, dass i für n\[C7]ix garantier',
Meine Fr\[G]eind' sog'n olle ``W\[B7]os'n, l\[Em]ossn, \[C7]
i maan, du f\[G]ührst di ganz schee d\[D7]eppert auf w\[B7]eg'n den Hos'n!''
B\[Em]itte, bitte, l\[D]oss mi n\[G]et so kn\[C]ian,
i mecht doch n\[G]ed mein' guaden R\[D7]uf verl\[G]ier'n.
\endverse

 

\beginchorus
Du bist die Bl\[G]ume aus d\[B7]em Gem\[Em]eindeb\[C]au, 
merkst' n\[G]icht wie ich sch\[D7]au,
wenn d\[G]u an mir vor\[C7]überschwebst,
du Bl\[G]ume aus d\[D7]em Gemeindeb\[G]au.
\endchorus


\beginverse 
Merkst du \[G]ned, wia i mi b\[B7]ei dir \[Em]eineh\[C]au,
weu du b\[G]ist für m\[D7]ich die \[G]Überfr\[C7]au,
komm, lass dich pfl\[G]ücken, du R\[D7]ose aus Stadl\[G]au! \[C7]
komm, lass dich pfl\[G]ücken, du R\[D7]ose aus St\[G]adlau! \[C7]
Komm, lass dich pfl\[G]ücken, du R\[D7]ose aus Stadl\[G]au!
\endverse

\endsong
